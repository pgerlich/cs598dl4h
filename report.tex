%
% File acl2021.tex
%
%% Based on the style files for EMNLP 2020, which were
%% Based on the style files for ACL 2020, which were
%% Based on the style files for ACL 2018, NAACL 2018/19, which were
%% Based on the style files for ACL-2015, with some improvements
%%  taken from the NAACL-2016 style
%% Based on the style files for ACL-2014, which were, in turn,
%% based on ACL-2013, ACL-2012, ACL-2011, ACL-2010, ACL-IJCNLP-2009,
%% EACL-2009, IJCNLP-2008...
%% Based on the style files for EACL 2006 by 
%%e.agirre@ehu.es or Sergi.Balari@uab.es
%% and that of ACL 08 by Joakim Nivre and Noah Smith

\documentclass[11pt,a4paper]{article}
\usepackage[hyperref]{acl2021}
\usepackage{times}
\usepackage{latexsym}
\renewcommand{\UrlFont}{\ttfamily\small}

% This is not strictly necessary, and may be commented out,
% but it will improve the layout of the manuscript,
% and will typically save some space.
\usepackage{microtype}

\aclfinalcopy % Uncomment this line for the final submission
%\def\aclpaperid{***} %  Enter the acl Paper ID here

%\setlength\titlebox{5cm}
% You can expand the titlebox if you need extra space
% to show all the authors. Please do not make the titlebox
% smaller than 5cm (the original size); we will check this
% in the camera-ready version and ask you to change it back.

% Content lightly modified from original work by Jesse Dodge and Noah Smith


\newcommand\BibTeX{B\textsc{ib}\TeX}

\title{Reproducibility Project Instructions for CS598 DL4H in Spring 2022}

\author{Paul Gerlich \\
  \texttt{gerlich2@illinois.edu}
  \\[2em]
  Group ID: 185, Paper ID: 80\\
  Presentation link: \url{https://www.youtube.com} \\
  Code link: \url{https://www.github.com/pgerlich/cs598dl4h}} 

\begin{document}
\maketitle

% All sections are mandatory.
% Keep in mind that your page limit is 8, excluding references.
% For specific grading rubrics, please see the project instruction.

\section{Introduction}
My goal is to reproduce the findings from the paper titled "Assertion Detection in Clinical Natural Language Processing: A Knowledge-Poor Machine Learning Approach". The paper focuses on labeling assertions in clinical notes, etc. as a form of pre-processing to remove extraneous information that might otherwise fool another machine learning algorithm that is processing those notes. The concept is that some clinical notes will mention diseases or symptoms that are NOT applicable to the current patient for some or other reason. The purpose of the model is to identify and remove these labels.

The interesting thing about this approach is that it works without being fed knowledge about what these assertions are. At the time of its writing, the most successful networks all had to be fed information about assertions.

\section{Scope of reproducibility}

The paper had two primary explorations. The primary claim was to create an Attention-based BiLSTM network that was not fed prior knowledge about assertions to attempt to achieve cutting edge accuracy (roughly 90-93 \%). Their claim was that a "knowledge poor" system could achieve comparable performance to a knowledge rich network in the assertion detection problem. 

The second claim was an experiment with word embedding algorithms in the embedding layer for the Att-BiLSTM model. Their findings in the second exploration did not meaningfully contribute to the primary claim. As such, I will take their findings as-is and use them in reproducing the primary claim.

\subsection{Addressed claims from the original paper}

Clearly itemize the claims you are testing:
\begin{itemize}
    \item A knowledge poor Att-BiLSTM network can achieve comparable accuracy to a Knowledge Rich network
    \item The PubMed+ word embedding would outperform a MIMIC III based word2vec word embedding when used in the Att-BiLSTM embedding layer
\end{itemize}


\section{Methodology}

There was no reference code for reproducing this network, so I am taking the description of the model architecture as outlined in the paper in an attempt to reproduce their findings. I have a single Mac-book Pro M1 with 16gb of RAM available to reproduce these findings. Considering that the dataset is incredibly small (on the order of 500 clinical notes with ~ 10k or less sentences) I do not believe that resource constraints will affect my ability to reproduce the results.

They do not mention any of their hyper-parameter values beyond the word embedding size so I had to do a bit of experimentation to come up with my final result.

\subsection{Model descriptions}
The paper describes the Att-BiLSTM model as follows:

Input Layer: Represents a single sentence that uses context markers to separate the target from the rest of the sentence.

Embedding Layer: Each input word is translated into a one-hot word embedding vector of size = 200, so the input is a sequence of size 200 vectors. Intention is to convert the words first into a one-hot vector of available vocab. Then to a word embedding of real numbers (size 200) to allow for word association and better contextual understanding.

BiLSTM Layer: Pass in embeddings to consider forward and backward context of each word, output is element-wise sum of word level features.

Attention Layer: Word level features are multiplied by weight vector and aggregated into sentence level feature vectors.

Output: Classification to one of 5 labels on sentence level feature vectors.

\subsection{Data descriptions}
The paper used data from the 2010 i2b2/VA NLP challenge on relation extraction. It further supplemented the dataset with the NegEx database which is associated with a preexisting model that tried to solve the same problem. This second dataset only had a subset of available assertions (negated or not negated) vs the 5 available labels in the i2b2 dataset. Nevertheless, this dataset helped to supplement the small amount of data available for this NLP challenge.

Today, the i2b2 dataset is contained within the n2c2 dataset and is managed by the Department of Biomedical Informatics (DBMI) at Harvard Medical School. I received access to the data upon a request to their staff.

\subsection{Hyperparameters}
TODO: 
Describe how you set the hyperparameters and what the source was for their value (e.g. paper, code or your guess). 

\subsection{Implementation}
The paper did not reference any repositories so I am trying to reproduce this experiment from scratch.

\subsection{Computational requirements}

I believe that I will need minimal time to train this network. I would think a quad core processor and a few compute hours at the maximum would do the trick. The challenging part of this experiment is determining the hyperparameters.

TODO: Actual consumption

\section{Results}
TODO: 
Start with a high-level overview of your results. Does your work support the claims you listed in section 2.1? Keep this section as factual and precise as possible, reserve your judgement and discussion points for the next ``Discussion'' section. 

Go into each individual result you have, say how it relates to one of the claims, and explain what your result is. Logically group related results into sections. Clearly state if you have gone beyond the original paper to run additional experiments and how they relate to the original claims. 

Tips 1: Be specific and use precise language, e.g. ``we reproduced the accuracy to within 1\% of reported value, that upholds the paper's conclusion that it performs much better than baselines.'' Getting exactly the same number is in most cases infeasible, so you'll need to use your judgement call to decide if your results support the original claim of the paper. 

Tips 2: You may want to use tables and figures to demonstrate your results.

% The number of subsections for results should be the same as the number of hypotheses you are trying to verify.

\subsection{Result 1}

\subsection{Result 2}

\subsection{Additional results not present in the original paper}
TODO: 
Describe any additional experiments beyond the original paper. This could include experimenting with additional datasets, exploring different methods, running more ablations, or tuning the hyperparameters. For each additional experiment, clearly describe which experiment you conducted, its result, and discussions (e.g. what is the indication of the result).

\section{Discussion}
TODO: 
Describe larger implications of the experimental results, whether the original paper was reproducible, and if it wasn’t, what factors made it irreproducible. 

Give your judgement on if you feel the evidence you got from running the code supports the claims of the paper. Discuss the strengths and weaknesses of your approach -- perhaps you didn't have time to run all the experiments, or perhaps you did additional experiments that further strengthened the claims in the paper.

\subsection{What was easy}
TODO: 
Describe which parts of your reproduction study were easy. E.g. was it easy to run the author's code, or easy to re-implement their method based on the description in the paper. The goal of this section is to summarize to the reader which parts of the original paper they could easily apply to their problem. 

Tips: Be careful not to give sweeping generalizations. Something that is easy for you might be difficult to others. Put what was easy in context and explain why it was easy (e.g. code had extensive API documentation and a lot of examples that matched experiments in papers). 

\subsection{What was difficult}
TODO: 
Describe which parts of your reproduction study were difficult or took much more time than you expected. Perhaps the data was not available and you couldn't verify some experiments, or the author's code was broken and had to be debugged first. Or, perhaps some experiments just take too much time/resources to run and you couldn't verify them. The purpose of this section is to indicate to the reader which parts of the original paper are either difficult to re-use, or require a significant amount of work and resources to verify. 

Tips: Be careful to put your discussion in context. For example, don't say ``the math was difficult to follow,'' say ``the math requires advanced knowledge of calculus to follow.'' 

\subsection{Recommendations for reproducibility}
TODO: 
Describe a set of recommendations to the original authors or others who work in this area for improving reproducibility.

\section{Communication with original authors}
TODO: 

Document the extent of (or lack of) communication with the original authors. To make sure the reproducibility report is a fair assessment of the original research we recommend getting in touch with the original authors. You can ask authors specific questions, or if you don't have any questions you can send them the full report to get their feedback.


\bibliographystyle{acl_natbib}
\bibliography{acl2021}

%\appendix



\end{document}

